%% start of file `template.tex'.
%% Copyright 2006-2012 Xavier Danaux (xdanaux@gmail.com).
%
% This work may be distributed and/or modified under the
% conditions of the LaTeX Project Public License version 1.3c,
% available at http://www.latex-project.org/lppl/.

% http://www.ctan.org/tex-archive/macros/latex/contrib/moderncv/


\documentclass[11pt,a4paper,sans]{moderncv}   % possible options include font size ('10pt', '11pt' and '12pt'), paper size ('a4paper', 'letterpaper', 'a5paper', 'legalpaper', 'executivepaper' and 'landscape') and font family ('sans' and 'roman')

% moderncv themes
\moderncvstyle{casual}                        % style options are 'casual' (default), 'classic', 'oldstyle' and 'banking'
\moderncvcolor{grey}                          % color options 'blue' (default), 'orange', 'green', 'red', 'purple', 'grey' and 'black'
%\renewcommand{\familydefault}{\sfdefault}    % to set the default font; use '\sfdefault' for the default sans serif font, '\rmdefault' for the default roman one, or any tex font name
%\nopagenumbers{}                             % uncomment to suppress automatic page numbering for CVs longer than one page

% character encoding
%\usepackage[utf8]{inputenc}                  % if you are not using xelatex ou lualatex, replace by the encoding you are using
%\usepackage{CJKutf8}                         % if you need to use CJK to typeset your resume in Chinese, Japanese or Korean

% enabling footnotes.
\usepackage[bottom,norule]{footmisc}

% adjust the page margins
\usepackage[scale=0.8]{geometry}
\setlength{\hintscolumnwidth}{3cm}           % if you want to change the width of the column with the dates
%\setlength{\makecvtitlenamewidth}{10cm}      % for the 'classic' style, if you want to force the width allocated to your name and avoid line breaks. be careful though, the length is normally calculated to avoid any overlap with your personal info; use this at your own typographical risks...

\setlength{\skip\footins}{1cm}

% personal data
\firstname{Savvas}
\familyname{Dalkitsis}
\title{Mobile Software Engineer}               % optional, remove the line if not wanted
\address{10 Orchid Street}{W12 0SY, London, Hammersmith}    % optional, remove the line if not wanted
\mobile{+44 7903432695}                     % optional, remove the line if not wanted
%\phone{+2~(345)~678~901}                      % optional, remove the line if not wanted
%\fax{+3~(456)~789~012}                        % optional, remove the line if not wanted
\email{savvas.dalkitsis@gmail.com}                          % optional, remove the line if not wanted
\homepage{www.savvasdalkitsis.com}                    % optional, remove the line if not wanted
\extrainfo{Skype: savvas.dalkitsis}            % optional, remove the line if not wanted
\photo[64pt][0.0pt]{picture}                  % '64pt' is the height the picture must be resized to, 0.4pt is the thickness of the frame around it (put it to 0pt for no frame) and 'picture' is the name of the picture file; optional, remove the line if not wanted
%\quote{Some quote (optional)}                 % optional, remove the line if not wanted

% to show numerical labels in the bibliography (default is to show no labels); only useful if you make citations in your resume
%\makeatletter
%\renewcommand*{\bibliographyitemlabel}{\@biblabel{\arabic{enumiv}}}
%\makeatother

% bibliography with mutiple entries
%\usepackage{multibib}
%\newcites{book,misc}{{Books},{Others}}

\newcommand{\superscript}[1]{$^{#1}$}
\newcommand{\urlfootnote}[2]{\footnotetext[#1]{\href{#2}{#2}}}

\newcommand{\personalproject}[4]{
\cvitem
  {#1}
  {#2\newline{}
  \href{#4}{#3} (\httplink{#4})}
}
  
\newcommand{\achievement}[5]{
\cventry
  {#1}
  {#2}
  {(\httplink{#3})}
  {#4}
  {}
  {#5}
  {}
}

\newcommand{\conference}[3]{
\cvitem{#1}{#2 (\httplink{#3})}
}
\newcommand\blfootnote[1]{%
  \begingroup
  \renewcommand\thefootnote{}\footnote{#1}%
  \addtocounter{footnote}{-1}%
  \endgroup
}



%----------------------------------------------------------------------------------
%            content
%----------------------------------------------------------------------------------
\begin{document}
%\begin{CJK*}{UTF8}{gbsn}                     % to typeset your resume in Chinese using CJK
%-----       resume       ---------------------------------------------------------
\makecvtitle

\section{Education}
\cventry{2003-2009}{Bachelor-Science (Mathematics)}{Aristotle University of Thessaloniki}{Greece}{}{}  % arguments 3 to 6 can be left empty
\cventry{2003}{A-Levels equivalent}{Drama}{Greece}{}{}

%\section{Master thesis}
%\cvitem{title}{\emph{Title}}
%\cvitem{supervisors}{Supervisors}
%\cvitem{description}{Short thesis abstract}
\section{Experience}
\cventry
  {04/2011-Now}
  {Mobile Software Engineer (Android)}
  {\href{www.shazam.com}{Shazam Entertainment}\superscript{1}}
  {London}
  {}
  { 
    Maintenance, design and implementation of new features on the Android client.\newline{}%
    Features implemented by me include:
    \begin{itemize}%
    \item A new, more robust, object oriented, audio recording sub-system.
    \item Rewriting of the 'tagging' (recognition process) sub-system using an async-state machine. 
    \item Preview playback for tagged songs using Android's media playback capabilities.
    \item Demonstrated, as part of the company's 'hack day', an NFC/Android Beam based way of sharing a user's tags, which made it to the final product.
    \item Mocked up various tablet designs for demonstration to partners.
    \item 3D lyrics visualizations for the product's LyricPlay\texttrademark  feature (written in C++)
    \end{itemize}
    Employed Test Driven Development to implement features in the client.\newline{}%
    Created a thin library to facilitate writing of Acceptance Tests in a Given-When-Then syntax that both benefit the product owner and QA due to it's English-like syntax and helps the development team define the domain of each feature more explicitly.\newline{}%
    Forked and contributed to android open source project \href{https://github.com/shazam/robolectric}{Robolectic}\superscript{2}\newline{}%
    \emph{Technologies used: \textbf{Java, Android, Maven, git, SQLite, TDD, C++, regex, JBehave}}\newline{}%
    \emph{Supporting software used: \textbf{IntelliJ, Bamboo, Jenkins, Jira, Fisheye}}
  }
  
\urlfootnote{1}{http://www.shazam.com}
\urlfootnote{2}{https://github.com/shazam/robolectric}

\cventry
  {01/2009-04/2011}
  {Software Engineer}
  {\href{www.hamptondata.com}{Hampton Data Services}\superscript{3}}
  {London}
  {}
  {  
    Maintenance and implementation of new features on the company flagship product "HDGeoscope", a GIS based information management solution targeted at companies in the Oil and Gas Industry.
    \newline{}%
    \begin{itemize}%
    \item Created an automated installation solution for "HDGeoscope" that manages the silent installation of many subcomponents such as MS SQL Server 2008 and its prerequisites as well as automatically setting it up for end use.
    \item Migrated source control from the outdated Microsoft Visual SourceSafe (VSS) to SVN and trained other developers on its usage.
    \item Modernised deployment process by setting up Continuous Integration and one click delivery from source to final product (DVD)
    \end{itemize}
  }
\urlfootnote{3}{http://www.hamptondata.com}

% hack to get space between page footnotes and bottom info panel
\blfootnote{}

% There is a page break and according to the creator of moderncv, there is no automatic way to break a cventry. So the solution
% he offers is to break the entry into separate cventry/cvitem parts.

\cvitem{}{
  \begin{itemize}%
  \item Completed sub-projects inside "HDGeoscope" include:
    \begin{itemize}%
    \item Creating a new, modern map navigation interface to replace the old one and bring it closer to the industry standards
          (changes include Google maps - like navigation by panning and interactive zooming)
    \item A geo-referencing system for placing images on the map
    \item Implementing a chart based information system that is integrated into the GIS map.
    \item Created a job scheduling sub-system that allows users across the network to create recurring jobs on the server.
    \item Several GUI improvements such as changing the look and feel of the application and implementing a more flexible layout
          system for its components.
    \end{itemize}
  \end{itemize}
  \emph{Technologies used: \textbf{Java, Swing, NSIS, Ant, SVN, cron, SQL, regex}}\newline{}%
  \emph{Supporting software used: \textbf{eclipse, MS SQL Server}}
}
  
\cventry
  {06/2008-01/2009}
  {Systems Administrator}
  {\href{www.eglobal.gr}{eGlobal Internet Stations}\superscript{4}}
  {Thessaloniki}
  {}
  {
    Responsibilities:%
    \begin{itemize}%
    \item Daily management of a network of 70 computer stations
    \item Hardware and software troubleshooting
    \item System maintenance
    \item Assistance to users
    \end{itemize}
  }
\urlfootnote{4}{http://www.eglobal.gr}

% hack to get space between page footnotes and bottom info panel
\blfootnote{}

\section{Programming/Scripting Languages}
\cvitem{Java}{8 years}
\cvitem{Pascal}{3 years}
\cvitem{NSIS}{1 year}
\cvitem{MS Visual Basic}{1 year}
\cvitem{C++}{basic knowledge}
\cvitem{SQL}{basic knowledge}
\cvitem{LaTeX}{basic knowledge}
\cvitem{Clojure, Scala}{novice/academic interest}


\section{Selected Personal Projects (available online)}
\personalproject
  {The Movie Database}
  {
    An Android front-end to "The Movie Database" at \href{http://www.themoviedb.org}{http://www.themoviedb.org}. 
    Currently, it has just over 10.000 downloads.
  }
  {Play Store Link}{https://play.google.com/store/apps/details?id=com.savvasdalkitsis.tmdb}
\personalproject
  {JTMDB}
  {A Java wrapper library for The Movie Database, an online repository for information about Movies.}
  {GitHub page}{https://github.com/savvasdalkitsis/JTMDB}
\personalproject
  {Joperties}
  {
    An extension to the Java Properties class. It exists to solve a common problem with Properties (all Properties are Strings).
    Joperties allows you to set and get Java Objects without worrying about serialization.
  }
  {Sourceforge page}{http://joperties.sourceforge.net}
\personalproject
  {JColorPicker}
  {
    An extremely simple desktop color picker. It allows you to select any color visible on your screen and copy its RGB code to the
    clipboard all done in one simple click and drag motion.
  }
  {Sourceforge page}{http://jcolorpicker.sourceforge.net}
\personalproject
  {JScreenGrabber}
  {A quick-to-use screen capture utility. It allows you to save parts of your screen to png files with one click.} 
  {Sourceforge page}{http://jscreengrabber.sourceforge.net}
\personalproject
  {InstallBuddy (Visual Basic .NET)}
  {
    InstallBuddy is designed to organize your installers into one easy-to-use interface. Its intended use is
    in an autorun CD/DVD.
  }
  {Sourceforge page}{http://installbuddy.sourceforge.net}
\personalproject
  {JMinesweeper}
  {A Java implementation of the well known puzzle game minesweeper.}
  {Sourceforge page}{http://jminesweeper.sourceforge.net}
  
\section{Relevant Achievements}
\achievement
  {06/07/2012}
  {Youtube/GTV hackathon}
  {http://ythacklon.appspot.com/}
  {London}
  {Attended and was part of a winning team in the Google weekend Youtube/GoogleTV hackathon.}
\achievement
  {2003}
  {Panhellenic Informatics Contest}
  {http://www.epy.gr}
  {Athens}
  {
    5\superscript{th} place, which also qualified me as the 1\superscript{st} runner-up for the International Olympiad in Informatics 2003
    \href{http://www.ioinformatics.org}{(http://www.ioinformatics.org)}
  }

\section{Relevant Conferences Attended}
\conference{27/06/2012}{GoogleIO}{https://developers.google.com/events/io}
\conference{13/02/2012}{JFokus}{http://www.jfokus.se}
\conference{06/10/2011}{Droidcon}{http://uk.droidcon.com}
\conference{07/10/2010}{Java2Days}{http://java2days.com}

\section{Languages}
\cvitem{Greek}{Native}
\cvitem{English}
  {
    Excellent command as certified by:
    \begin{itemize}
    \item University of Cambridge - Certificate of Proficiency in English
    \item University of Michigan - Certificate of Proficiency in English
    \end{itemize}
  }

\section{Personal Interests}
\cvlistitem{New developments in digital technology.}
\cvlistitem{3D modeling (3 years of experience with Autodesk 3D studio MAX)}
\cvlistitem{Film post production (involved in editing, special effects and digital processing of 2 short films)}
\cvlistitem{Astronomy}
\cvlistitem{Natural Sciences literature}
\cvlistitem{Climbing}

% Publications from a BibTeX file without multibib
%  for numerical labels: \renewcommand{\bibliographyitemlabel}{\@biblabel{\arabic{enumiv}}}
%  to redefine the heading string ("Publications"): \renewcommand{\refname}{Articles}
%\nocite{*}
%\bibliographystyle{plain}
%\bibliography{publications}                   % 'publications' is the name of a BibTeX file

% Publications from a BibTeX file using the multibib package
%\section{Publications}
%\nocitebook{book1,book2}
%\bibliographystylebook{plain}
%\bibliographybook{publications}              % 'publications' is the name of a BibTeX file
%\nocitemisc{misc1,misc2,misc3}
%\bibliographystylemisc{plain}
%\bibliographymisc{publications}              % 'publications' is the name of a BibTeX file

\clearpage
%-----       letter       ---------------------------------------------------------
% recipient data
\recipient{Company Recruitment team}{Company, Inc.\\123 somestreet\\some city}
\date{January 01, 1984}
\opening{Dear Sir or Madam,}
\closing{Yours faithfully,}
\enclosure[Attached]{curriculum vit\ae{}}     % use an optional argument to use a string other than "Enclosure", or redefine \enclname
\makelettertitle

Lorem ipsum dolor sit amet, consectetur adipiscing elit. Duis ullamcorper neque sit amet lectus facilisis sed luctus nisl iaculis. Vivamus at neque arcu, sed tempor quam. Curabitur pharetra tincidunt tincidunt. Morbi volutpat feugiat mauris, quis tempor neque vehicula volutpat. Duis tristique justo vel massa fermentum accumsan. Mauris ante elit, feugiat vestibulum tempor eget, eleifend ac ipsum. Donec scelerisque lobortis ipsum eu vestibulum. Pellentesque vel massa at felis accumsan rhoncus.

Suspendisse commodo, massa eu congue tincidunt, elit mauris pellentesque orci, cursus tempor odio nisl euismod augue. Aliquam adipiscing nibh ut odio sodales et pulvinar tortor laoreet. Mauris a accumsan ligula. Class aptent taciti sociosqu ad litora torquent per conubia nostra, per inceptos himenaeos. Suspendisse vulputate sem vehicula ipsum varius nec tempus dui dapibus. Phasellus et est urna, ut auctor erat. Sed tincidunt odio id odio aliquam mattis. Donec sapien nulla, feugiat eget adipiscing sit amet, lacinia ut dolor. Phasellus tincidunt, leo a fringilla consectetur, felis diam aliquam urna, vitae aliquet lectus orci nec velit. Vivamus dapibus varius blandit.

Duis sit amet magna ante, at sodales diam. Aenean consectetur porta risus et sagittis. Ut interdum, enim varius pellentesque tincidunt, magna libero sodales tortor, ut fermentum nunc metus a ante. Vivamus odio leo, tincidunt eu luctus ut, sollicitudin sit amet metus. Nunc sed orci lectus. Ut sodales magna sed velit volutpat sit amet pulvinar diam venenatis.


\makeletterclosing

%\clearpage\end{CJK*}                         % if you are typesetting your resume in Chinese using CJK; the \clearpage is required for fancyhdr to work correctly with CJK, though it kills the page numbering by making \lastpage undefined
\end{document}


%% end of file `template.tex'.
